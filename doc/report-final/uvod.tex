\chapter{Uvod}
V zadnjem desetletju se je na podro�ju komunikacij 
naredil ogromen korak. Razvoj telekomunikacij in 
mobilnih naprav nam je omogo�il danes poleg prenosa govora 
tudi po�iljanje in sprejemanje sporo�il, prenos ve�predstavnostnih 
vsebin, integrirano navigacijo in podobno. Na podro�ju osebnih ra�unalnikov 
pa je poleg elektronske po�te zacvetelo t.i. neposredno sporo�anje (eng. instant messaging).

Na�a seminarska naloga je vklju�evala razvoj aplikacije za mobilne telefone 
za neposredno sporo�anje. Ker je dana�njih najbolj uporabljenih protokolov 
za sporo�anje ve�, je bilo smiselno ustvariti aplikacijo, ki podpira ve� 
protokolov hkrati. Uspe�no smo implementirali protokole Jabber, MSN in ICQ. 
Aplikacijo smo napisali za mobilne naprave, ki imajo name��eno J2ME. Danes je 
to �e ve�ina vseh mobilnih telefonov mlaj�ih od petih let.

\section{Razvojno okolje}
Za razvoj na�e aplikacije smo uporabili okolje Java 2 Mobile Edition 
(http://java.sun.com/j2me). Podrejali smo se MIDP 2.0 in CLDC 1.1 standardom za mobilne naprave. 
Kot IDE smo uporabljali Eclipse (http://www.eclipse.org) in NetBeans (http://www.netbeans.org).
Projekt je odprt in se nahaja na Berlios stre�niku 
(http://developer.berlios.de/projects/jimmyim). Vsa zgodovina razvoja se lahko dostopa 
preko SVN drevesa (http://svn.berlios.de/wsvn/jimmyim).
Aplikacijo smo uspe�no testirali na telefonih:
\begin{itemize}
\item Siemens M65
\item Nokia 3220
\item Nokia 6610
\end{itemize}

\section[Pojmi]
Takoj�nje oz. neposredno sporo�anje je izraz za Instant messaging oz. IM v angle��ini.

JimmyIM (uradno ime), JIMMY (tehni�no ime), Jimmy (ime razreda) ali jimmy (ime paketa) so izrazi za ime na�e aplikacije za takoj�nje
 sporo�anje.
Razvijalec je v tej dokumentaciji mi�ljen programer, ki razvija aplikacijo.

Uporabnik je v tej dokumentaciji mi�ljen kot kon�ni uporabnik aplikacije na mobilnih telefonih.

\section{Licenca}
JimmyIM je last izklju�no njegovih avtorjev navedenih v datoteki AUTHORS in na 
naslovnici tega poro�ila. Celoten projekt je odprt. Raz�irja se lahko pod pogoji 
licence GNU GPL v2 (http://www.gnu.org/licenses/gpl.txt).

Slike protokolov (Jabber �arnica, MSN metulj in ICQ cvet) so last KDE programa 
za neposredno sporo�anje, Kopete (http://kopete.kde.org).