\chapter{Osnovni razredi}
JimmyIM je razbit na ve� paketov:

\begin{itemize}
\item \textbf {jimmy}: Temeljni razredi, Ra�uni, Kontakti
\item \textbf {jimmy.net}: Razredi namenjeni povezovanju na IM stre�nike.
\item \textbf {jimmy.util}: Splo�ni algoritmi potrebni za komunikacijo z IM stre�niki (npr. MD5)
\item \textbf {jimmy.ui}: Uporabni�ki vmesnik
\item \textbf {jimmy.jabber}: Jabber protokol
\item \textbf {jimmy.msn}: MSN protokol
\item \textbf {jimmy.icq}: ICQ protokol
\end{itemize}

\section{Ra�uni}
\textbf{jimmy.Account}
Razred je namenjen hranjenju lastnosti ra�una. Podprti ra�uni:
\begin{itemize}
\item Jabber
\item MSN
\item ICQ
\end{itemize}

Vsak ra�un vsebuje naslednje lastnosti:
\begin{itemize}
\item Uporabni�ko ime (userName\_)
\item Geslo (password\_)
\item Ciljni stre�nik (po �elji) (server\_)
\item Ciljna vrata (po �elji) (port\_)
\item Samodejna prijava (autoLogin\_)
\end{itemize}

\section{Protokoli}
\textbf{jimmy.Protocol}
Vsak ra�un potrebuje za delovanje uporabo razreda Protocol. Abstraktni razred Protocol je v splo�nem namenjen prijavi, po�iljanju in sprejemanju sporo�il in stanj uporabnikov, odjavi, enkripciji ipd.

Glavne metode:
\begin{itemize}
\item constructor(Jimmy)
\item boolean login(Account)
\item void sendMsg(String)
\item void logout()
\end{itemize}

Implementacije abstraktnega razreda Protocol:
\begin{itemize}
\item JabberProtocol
\item MSNProtocol
\item ICQProtocol
\end{itemize}

Ve� o posameznih zna�ilnostih protokolov v poglavju Protokoli.

\section{Kontakti}
\textbf{jimmy.Contact}
Razred je namenjen hranjenju lastnosti kontakta. Kontakt je vsak uporabnik nekega protokola s katerim lahko komuniciramo. Kontakt vsebuje naslednje lastnosti:
\begin{itemize}
\item Uporabni�ko ime za doti�en protokol (userID\_)
\item Prikazano ime (screenName\_)
\item Stanje - na voljo, zaposlen, stran od ra�unalnika, nepovezan (status\_)
\item Ime skupine (groupName\_)
\item Referenca na protokol h katermu pripada (protocol\_)
\end{itemize}
