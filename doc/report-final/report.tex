\documentclass[a4paper,11pt]{report}
%
%--------------------   start of the 'preamble'
%
%\usepackage{graphicx,amssymb}  %% ... with default font
\usepackage[slovene]{babel} % slovenski izpis naslovov in "c"s"z 
\usepackage[latin2]{inputenc} % vnos 蹾 po ISO Latin 2 
%\usepackage{graphicx,amssymb,mathptmx}  %%% times roman font
%\usepackage{graphicx,amssymb,mathpazo}   %%% another font
%\usepackage{graphicx,amssymb,newcent}   %%% another font
%\usepackage{graphicx,amssymb,cmbright}   %% yet another
%%%%%%%%%     more fonts in Chap 7 of the 'Companion'
%
%\newcommand\etc{\textsl{etc}}
%\newcommand\eg{\textsl{eg.}\ }
%\newcommand\etal{{\em et al.}}
%\newcommand\Quote[1]{\lq\textsl{#1}\rq}
%\newcommand\fr[2]{{\textstyle\frac{#1}{#2}}}
%
%---------------------   end of the 'preamble'
%
\begin{document}
%-----------------------------------------------------------
\title{JimmyIM - J2ME klient za takoj�nje sporo�anje}
\author{\textbf{Avtorji (v abecednem vrstnem redu):}\\Matev� Jekovec (Jabber)\\Zoran Mesec (MSN)\\Dejan Sakel�ak (ICQ)\\Janez Urevc (uporabni�ki vmesik)}
\maketitle
%-----------------------------------------------------------
\begin{abstract}\centering
JimmyIM je J2ME aplikacija za takoj�nje sporo�anje. Podpira naslednje protokole: 
Jabber, MSN in ICQ. Aplikacija je namenjena za uporabo na vseh mobilnih napravah, 
ki podpirajo Javo\footnote{Java in J2ME sta za��iteni blagovni znamki podjetja 
Sun Microsystems}.
\end{abstract}
%-----------------------------------------------------------
\tableofcontents
%-----------------------------------------------------------
\chapter{Uvod}
V zadnjem desetletju se je na podro�ju komunikacij
naredil ogromen korak. Razvoj telekomunikacij in
mobilnih naprav nam je omogo�il danes poleg prenosa govora
tudi po�iljanje in sprejemanje sporo�il, prenos ve�predstavnostnih
vsebin, integrirano navigacijo in podobno. Na podro�ju osebnih ra�unalnikov
pa je poleg elektronske po�te zacvetelo t.i. neposredno sporo�anje (eng. instant messaging).

Na�a seminarska naloga je vklju�evala razvoj aplikacije za mobilne telefone
za takoj�nje sporo�anje. Ker je dana�njih najbolj uporabljenih protokolov
za sporo�anje ve�, je bilo smiselno ustvariti aplikacijo, ki podpira ve�
protokolov hkrati. Uspe�no smo implementirali protokole Jabber, MSN in ICQ.
Aplikacijo smo napisali za mobilne naprave, ki imajo name��eno J2ME. Danes je
to �e ve�ina vseh mobilnih telefonov mlaj�ih od petih let.

\section{Razvojno okolje}
Za razvoj na�e aplikacije smo uporabili okolje Java 2 Mobile Edition
(http://java.sun.com/j2me). Podrejali smo se MIDP 2.0 in CLDC 1.1 standardom za mobilne naprave.
Kot IDE smo uporabljali Eclipse (http://www.eclipse.org) in NetBeans (http://www.netbeans.org).
Projekt je odprt in se nahaja na Berlios stre�niku
(http://developer.berlios.de/projects/jimmyim). Vsa izvorna koda in njena zgodovina se lahko dostopa
preko SVN drevesa (http://svn.berlios.de/wsvn/jimmyim).
Aplikacijo smo uspe�no testirali na telefonih:
\begin{itemize}
\item Siemens M65
\item Nokia 3220
\item Nokia 6610
\end{itemize}

\section{Pojmi}
Takoj�nje oz. neposredno sporo�anje je izraz za Instant messaging oz. IM v angle��ini.

JimmyIM (uradno ime), JIMMY (tehni�no ime), Jimmy (ime razreda) ali jimmy (ime paketa) so izrazi za ime na�e aplikacije za takoj�nje
 sporo�anje.
Razvijalec je v tej dokumentaciji mi�ljen programer, ki razvija aplikacijo.

Uporabnik je v tej dokumentaciji mi�ljen kot kon�ni uporabnik aplikacije na mobilnih telefonih.

\section{Licenca}
JimmyIM je last izklju�no njegovih avtorjev navedenih v datoteki AUTHORS in na
naslovnici tega poro�ila. Celoten projekt je odprt. Raz�irja se lahko pod pogoji
licence GNU GPL v2 (http://www.gnu.org/licenses/gpl.txt).

Slike protokolov (Jabber �arnica, MSN metulj in ICQ cvet) so last KDE programa
za neposredno sporo�anje, Kopete (http://kopete.kde.org).
\chapter{Opis protokolov}
\section{Jabber--XMPP}
\subsection{Splo�no}
Jabber (http://www.jabber.org) je najbolj raz�irjen odprt protokol za neposredno 
sporo�anje. Temelji na XML sintaksi. Osnovni, jedrni del je dokaj enostaven z obilico 
prostora za raz�iritve. Trenutno �e obstaja veliko odjemalcev (MirandaIM za Windows, 
Kopete za KDE, Gaim napisan v gtk+, deluje pod veliko OS, Bombus za J2ME in mnogi drugi), 
kot tudi stre�nikov (najbolj znan je v Javi napisan Wildfire). Ravno v tem �asu tudi 
Google razvija svoj Google Chat, ki uporablja XMPP kot temeljno sintakso za neposredno 
sporo�anje, za prenos zvoka pa algoritem Speex (http://www.speex.org), vgnezden v XMPP 
sintakso. Naj omenim, da tudi danes najbolj popularen program za pogovor preko interneta, 
Skype (http://www.skype.org), uporablja Speex algoritem za kompresijo zvoka. Problem je le 
v zaprtosti protokola, saj je Skype lastni�ki komercialni program. Upajmo, da bo Googleova 
tradicionalna naklonjenost odprtokodni skupnosti tudi tokrat obrodila sadove.

Prav odprtost in enostavnost dajeta Jabberju prednost pred ostalimi protokoli. Jabber 
ni centraliziran, ampak obstaja celotna mre�a med seboj povezanih razli�nih stre�nikov 
(npr. jabber.org, gristle.org, predoslje.org). Vsak uporabnik ima svoje unikatno ime v 
obliki ime@stre�nik.

Jabber prav tako podpira razli�no enkripcijo pri prenosu sporo�il. Poleg standardne SASL 
in TLS povezave je na voljo tudi vi�jenivojska mo�na SHA1 (temelje�a na klju�ih oz. 
certifikatih) enkripcija podatkovenga dela XML kitice. Enkriptirani podatki se prena�ajo med 
odjemalcem in stre�nikom, kot tudi med stre�niki.

Prav tako zanimiva lastnost Jabber stre�nikov je t.i. prehod (gateway) med Jabberjem in 
ostalimi protokoli, torej da pretvorbo sporo�il in stanj uporabnikov v druge protokole 
izvaja Jabber stre�nik samodejno (npr. za Wildfire obstaja kopica raz�iritev za razli�ne 
protokole). Odjemalec mora poznati le Jabber sintakso.

\subsection{Zna�ilnosti}
\begin{itemize}
\item odprt protokol, jasna specifikacija
\item XML sintaksa
\item mo�na enkripcija celotno pot od odjemalca do odjemalca
\item hranjenje sporo�il na stre�niku, �e uporabnik trenutno ni dosegljiv
\item mo�nost raz�iritev obstoje�e sintakse na druga podro�ja (npr. VoIP)
\item konferen�na povezava
\item prenos datotek
\end{itemize}

\subsection{Implementacija}

\textbf{jimmy.jabber.JabberProtocol}

Osnovni razred, ki predstavlja protokol Jabber je JabberProtocol. Po stvaritvi objekta 
se pokli�e metoda login(), ki aktivira nit (ki ves �as preverja stanje prejetih 
Jabber sporo�il), metoda pa vrne true ob uspe�ni prijavi ali false ob napaki. JabberProtocol 
se privzeto prijavi na stre�nik, ki je podan ob uporabni�kem imenu. Za po�iljanje 
sporo�il se uporablja standardna metoda sendMsg(), za odjavo pa logout().

JabberProtocol ne uporablja splo�nega DOM ali SAX raz�lenjevalnika (eng. parser), 
ampak smo zaradi hitrosti in optimizacije kode napisali specifi�ni raz�lenjevalnik 
JabberParseXML namenjen prav branju sporo�il, stanja kontaktov, spreminjanju 
lastnosti itd. za XMPP protokol.

JabberProtocol ne podpira �e nobene enkripcije linije, kot tudi ne enkripcije XML kitic. 
Vsi podatki se prena�ajo v �istem tekstovnem na�inu v UTF-8 kodnem naboru.

\subsection{Podprte zna�ilnosti}
\begin{itemize}
\item podpora ve�ih Jabber ra�unov hkrati
\item za�etek pogovora s poljubnim kontaktom na poljubnem Jabber stre�niku, �e je pobudnik JIMMY uporabnik
\item za�etek pogovora s poljubnim kontaktom na poljubnem Jabber stre�niku, �e je kontakt pobudnik pogovora
\item prikaz stanj kontaktov (online, offline, busy, away)
\item urejanje lastnosti kontakta in dodeljevanje skupinam
\item dodajanje kontaktov, �e je pobudnik JIMMY uporabnik
\item dodajanje kontaktov, �e je pobudnik kontakt
\item brisanje kontaktov
\end{itemize}

\section{MSNP}
MSN Messenger je program za neposredno sporo�anja, sicer produkt podjetja Microsoft. 
Na trgu je vse od leta 1999, od takrat je do�ivel veliko izbolj�av in pohitritev. Program 
za komunikacijo uporablja tekstovni protokol, ki se imenuje MSN Messenger protokol (z angle�ko 
kratico MSNP). Program sodi med �tiri najpriljubljenej�e programe za neposredno sporo�anje, 
najbr� lahko re�emo, da je med slovenskimi uporabniki kar najpogostej�i. Razlog, da smo v 
Jimmy-iju implementirali MSN protokol je  pogostosti uporabe med slovenskimi uporabniki. 

Program MSN Messenger, ki deluje v okolju Windows je t.i. klient, oziroma program, ki te�e 
na uporabnikovem ra�unalniku, za delovanje pa potrebuje oddaljeni stre�nik s katerim ves 
�as delovanja komunicira. Program lokalno ne hrani nobenih podatkov o seznamu kontaktov 
in skupin, temve� informacije dobiva preko povezave s centralnim stre�nikom. Pri komunikaciji 
med stre�nikom in MSN Messengerjem se uporablja MSN protokol (MSNP). Program  MSN Messenger 
je bil izhodi��e za implementacijo MSN protokola na Jimmy-ju. Sorodnosti med programoma 
so o�itne, mo�no je opaziti tudi podobnosti z drugimi protokoli in programi za neposredno 
sporo�anje. 

Program na osebnem ra�unalniku se imenuje MSN Messenger ``klient''. Njegova naloga je, da 
se preko interneta pove�e na MSN Messenger stre�nik. Gledano �ir�e, vsaka menjava informacije 
med razli�nimi klient programi, poteka preko centralnega stre�nika, mimo te povezave sporo�anje 
med uporabniki ni mo�no. Med delovanjem bo torej na� program ``govoril'' oz. po�iljal 
informacije do centralnega stre�nika, ki bo nato poskrbel, da paketi pridejo do pravega 
naslovnika. Kakorkoli �e, obstajajo tudi dolo�eni ukazi protokola za katere velja, da 
jih centralni stre�nik ne obdela, temve� samo posreduje do naslovnika -- ukazi se nato 
procesirajo pri naslovniku (o tak�nih ukazih ve� kasneje). 

\begin{quotation}
Microsoft recommends MSN Messenger for most Windows users, except on Windows XP, where 
Windows Messenger is bundled with the operating system. Other people and companies have 
written "third party" MSN Messenger clients. You can see some of them listed on the 
resources page. MSN Messenger is generally considered the de-facto standard client, 
and most clients take their lead from its behaviour, so it's referred to as 
``the official client'' on this site.
\end{quotation}

Microsoft priporo�a program MSN Messenger vsem uporabnikom operacijskega sistema MS 
Windows kot nekak�en de facto standard za neposredno sporo�anje (na Windows XP je program 
vklju�en v osnovnem namestitvenem paketu), kar bi lahko ozna�ili za manj�o la�. S tem si 
podjetje seveda zagotovi zvestobo uporabnikov in nadaljno uporabo njihovih storitev vklju�no 
z MSN protokolom. Politika tega podjetja je, kot �e najbr� ves �as obstoja, predvsem 
pridobitni�ka. Ta miselnost  zajema tako denarno plat ra�unalni�tva, kot tudi vpliva na 
�ivljenjski slog uporabnikov, ki jih je  Microsoft ``podjarmil'' s svojimi produkti. 
Seveda za vsako izre�eno mislijo obstaja ozadje in tudi tu verjetno ne gre iskati izjem.

Zgolj pljuvati in blatiti ime Microsoft na ra�un njihove kapitalisti�ne usmerjenosti 
in monopolizma bi bilo slaboumno in zato tega v tem besedilu ne bom po�el. 
S stali��a ra�unalni�kega in�enirja pa lahko izrazim dolo�ene kritike do politike podjetja. 
Kot sem �e napisal, se za vsako izjavo skriva ozadje. MSN Messenger ni naklju�no edini 
uradni program za sporo�anje preko MSN protokola. Izbran je bil, ker deluje samo na 
o.s. MS Windows. kot t.i. ``MS Windows service'' ali druga�e, je ena izmed funkcionalnosti 
o.s. MS Windows. S to potezo si je Microsoft monopolisti�no pridr�al uporabnike in pravico 
do uporabe svojega protokola. Vsak drug program, ki uporablja MSN protokol (angle�ki 
izraz je ``third party client'') je avtomati�no neza�elen in Microsoft se z vsako novo 
verzijo protokola vztrajno trudi, da bi tak�ne programe odgnal. V proces preverjanja 
prisotnosti, so tako na primer v verzijo 11 vgradili kompleksen algoritem, ki je izredno 
zahteven za implementacijo in naj bi slu�il ravno temu namenu. V krvi ra�unalni�kih 
in�enirjev je, da tak�ne poteze grajamo. Potreba po standardizaciji tako na internetu kot 
na podro�ju programske opreme je danes zaradi porasta razli�nih tehnologij, ki temeljijo 
na razli�nih jedrih, nujnej�a kot kdajkoli prej in za�eleno je, da so standardi odprte 
narave (bodisi odprtokodni ali pa potrjeni od organizacij za standardizacijo kot je npr. ISO), 
ravno zaradi prenosljivosti in dostopnosti. Nasploh je zadnja leta v ra�unalni�tvi �utiti 
tendenco po ve� platformnih re�itvah in resno konkurenco odprto-kodnih re�itve. Microsoftove 
odlo�itve vejijo v nasprotno smer. Tu naj omenim, da je bila dokumentacija za MSN protokol, 
ki je dostopna �ir�i javnosti, spisano s pomo�jo t.i. ``reverse engineering-a''. To dejstvo 
dokazuje moje trditve, da �eli Microsoft odvrniti ostale razvijalce programske opreme od 
uporabe MSN protokola. 

Na globalnem trgu obstaja med operacijskimi sistemi velik boj za uporabnike. Razlog zakaj 
je Microsoft najuspe�nej�i predvsem pri doma�ih uporabnikih je pretanjeno  



V Jimmy-ju smo tako implementirali verzijo MSN protokola 9 in 10, kar nam omogo�a, da na 
v enem programu zdru�imo ve� t.i. klientov. Jimmy

\section{ICQ}
\subsection{Protokol --  ``OSCAR''}
OSCAR (Open System for ComunicAtion in Realtime) je binarni protokol za instantno sporocanje, 
ki je leta 2001 nadomestil takratni ICQ-jev TOC protokol. Neglede na njegovo ime, je protokol 
zaprt in specifikacijo si lasti AOL. Do ``odprtih'' in nenatan�nih specifikacij protokola 
so se nekateri prebili z obratnim in�iniringom. OSCAR se trenutno uporablja za ICQ in AIM 
komunikacijska sistema. Sistema uporabljata skupni nabor enot protokola, a vsak od njiju 
ima tudi entitete, ki niso skupne skupne obema.

Komunikacija med stre�niki in klienti poteka v paketni obliki. OSCAR vsebuje tri nivoje 
paketov in sicer FLAP, SNAC in TLV. V�asih se v FLAP ali SNAC paketu nahajajo tudi goli 
podatki, ki niso v paketni obliki.

\begin{enumerate}
\item FLAP: osnovni nivo, preko katerega se prena�ajo vsi podatki. Vsebuje informacijo 
  o kanalu (tipu podatkov, v njemu), dol�ini podatkov, zaporedno �tevilko paketa ter podatke.
  \begin{enumerate}
  \item Kanal 1 -- "Hello": je namenjen osnovni avtentifikaciji.
  \item Kanal 2 -- "SNAC": samo preko tega kanala se prena�ajo SNAC "paketi".
  \item Kanal 3 -- "Error": ko se prejme nepravilen paket oziroma pride do napake 
    na FLAP nivoju, se klientu/stre�niku po�lje informacija o napaki preko tega kanala.
  \item Kanal 4 -- "Disconnect": ko prejmemo paket s tem kanalom, moramo prekiniti povezavo.
  \item Kanal 5 -- "Keepalive": da stre�niku vedeti, da je klient �evedno �iv.
  \end{enumerate}
\item SNAC: storitveni nivo, ki natan�neje definira vsebino paketa. V glavi je zabele�ena 
dru�ina stortve, ukaz, zastavice ter referenca zahtevka. Poznanih je pribli�no 23 dru�in 
storitev od katerih ICQ uporablja pribli�no polovico.
\item TLV: ime tega nivjo je kratica za, Type Lenght Value, kar nam pri�epne, da so v 
teh paketih dejanski podatki. V glavi paketa je zapisano ime podatka ter njegova dol�ina.
\end{enumerate}

Storitev, ki so mo�ne v ICQ sistemu je veliko. Od obi�ajnega pogovarjanja in sledenja 
kontaktom do reklam itd.

\subsection{Implementacija}
Implementacija protokola je deljena na dve stopnji. Na stopnjo prijave in stopnjo avtomata. 
Stopnja prijave je trenutno v stati�ni obliki, kar ni najbolj priro�no ob izjemah, a 
problem pomanjkanja dokumentacije je tukaj o�iten. Za bolj�o implementacijo prijavne sekvence 
bi bilo potrebno poznavanje delovanja stre�ni�kih sistemov na AOL-u. Avtomatski del poganja 
neskon�na zanka, ki dr�i teko�o nit pri �ivljenju. V zanki se vsak cikel preverja izravnalnik 
toka, �e vsebuje nove pakete.

Paketi v sistem pridejo v obliki tabele bajtov ter se pretvorijo v objekte na katerih so 
definirane dolo�ene operacije. Po pretvarjanju (rezanju glav paketov) se paket tolma�i. 
Tolma� pakete spu��a skozi switch stavek, ko se ujame se iz njega izvle�ejo podatki in 
glede na njih izvede neko akcijo.

Po�iljanje paketov izgleda podobno, le da v ve�ini primerov njihovo po�iljanje povzro�i 
klic metode iz jimmyjevega jedra. Paket se pripravi z generiranjem novega osnovnega paketa, 
�e bo paket vseboval SNAC podatke, bo potrebno nastaviti tudi parametre SNAC glave. Ker je 
vsebina paketa lahko zelo razli�na, je potrebno tudi to nastaviti po dolo�enem postopku. 
Podatke, ki so ``prosto plavajo�i'' se nastavi kot vsebino v obliki tabele bajtov. �e �elimo 
pripeti tudi TLV paket, moramo ustvariti nov TLV objekt in mu nastaviti vse koli�ine ter ga 
dodati v listo TLV paketov v na�em osnovnem paketu. Ko so vsi podatki v njem, se pokli�e 
metodo, ki vrne paket v tabeli bajtov in jo po�lje v tok.

Implementacija protokola je za na� primer omejena samo na ICQ sistem. Razporejena je �ez ve� 
javinih razredov: Protocol, ICQProtokol, ICQPackage, ICQTlv, ServerHandler, ICQConnector, 
ICQContact, Utils ter ostale razrede za interakcijo z jedrom. ICQProtokol je raz�iritev 
abstraktnega razreda Protokol, ki je gradnik implementacij vseh protokolov v Jimmyju vsebuje 
poleg osnovega nabora atributov ter metod �e dolo�ene za ICQ specifi�ne kot so metode za 
tolma�enje, metode za spreminjanje stanj v instanci ter metode za branje vrednosti iz jedra. 
ICQPackage je splo�na specifi�na podatkovna struktura, ki ima definirane operacije zase. 
Vsebuje dolo�ene kolekcije ter metode za vra�anje vsebine in nastavljanje vsebine. ICQTlv je 
namenjena generiranju TLV paketov. Definirana je sama zase, da lahko nje naredimo ve� instanc. 
Vsebuje operacije ter strukture nad katerimi izvajamo te operacije. ICQConnector je raz�iritev 
ServerHandler razreda, ki omogo�a enostavnej�e branje paketov. ServerHanler, namre�, vra�a 
vse kar se je spravilo v bralni izravnalnik, kar pa se lahko odra�a z ve�jim �tevilom FLAP 
paketov. S klicem metode za vra�anje paketa v ICQConnectorju tako to informacijo razre�emo 
na pakete in shranimo v kolekciji ter ob vsakem ciklu zankepreverimo �tevilo paketov v njej 
ali preberemo novo iz izravnalnika. ICQContact je samo raz�iritev splo�nega razreda Contact. 
Ima dodatna polja za hranjenje informacij os ID-ju skupine in samega kontkata v SSI 
(Server Side Information) listi.

Ker so v OSCAR-ju vsi podatki v binarni obliki potrebujemo pripomo�ke za pretvorbe med 
razli�nimi tipi. V paketu Utils tako najdemo stati�ne metode, za pretvarjanje iz tabel 
bajtov v short, int ali long primitive ter obratno.

\subsection{Problemi}
Na poti do uspe�nega delovanja sem se sre�al z velikim �tevilom problemov, ki so mi od�rli 
marsikateri dan �ivljenja. Problemi so sicer nastali zaradi pomanjkljive in nepopolne 
dokumentacije protokola, a so v veliki meri nastale tudi zaradi slabega na�rtovanja. 

Najhuj�i problem je bil z razumevanjem prijavne sekvence, ker v nobeni dokumentaciji ni 
pisalo kako to�no prijava izgleda. V dolo�enih dokumentacijah je bila prijava celo pretirano 
napihnjena in zakomplicirana. Pri tem problemu so me re�ile �e narejene implementacije. Z 
orodjem za bele�enje mre�nih paketov (ethereal) sem bele�il komunikacijo s stre�nikom in na 
tak na�in spraskal nekako pravilna zaporedja. Ko sem pa �elel dolo�eno funkcionalnost 
izklopiti zaradi omejitve na GSM aparatih sem do�ivel tudi spremembo zaporedja, kar je 
vplivalo tudi na motivacijo. 

Problem je nastal tudi pri razumevanju vsebine paketa s kontaktno listo in skupinami, ker 
sem zaradi druga�nega prijavnega zaporedja v njem dobil �e dolo�ene neznane podatke, ki so 
mi podrli logiko.

Del, ki �e ni popolnoma implementiran je dodajanje in odstranjevanje kontaktov, kajti v 
nobeni dokumentaciji ne pi�e kako se to po�ne in je potrebno vse bele�iti in iskati izjeme 
ter ugotavljati delovanje.

Sre�al sem se tudi s problemom pretvarjanja sprejetega teksta iz Unicode tabele bajtov v 
Unicode tekst. Tega problema �e nisem odpravil, bo pa nujno to realizirati, kajti nesmiselno 
je sprejemanje dvojne koli�ine podatkov, �e jih potem ne moremo uporabit.

\subsection{Lastno mnenje}
Med implementacijo sem se velikokrat spra�eval, zakaj je ta protokol tak kot je. Ugotovitve 
pa so bile, da je bil zastavljen pre�iroko. �e pogledamo dolo�ene lastnosti bomo opazili 
nesmisle za trenutno obliko. Kot primer vzemimo polje v SNAC glavi, ki nam povestoritvene 
dru�ine. To polje je veliko 2 bajta, vseh dru�in pa je zgolj 23. Taka zasnova nas pripelje 
do velikih redundanc v paketih. 

Veliko sem se spra�eval tudi zakaj je mo�no imeti podatke tudi zunaj TLV paketov. TLV 
paket ima svoj namen in ne vidim razloga zakaj ga ne uporabit. �e vzamemo kot primer paket 
s kontaktno listo, vsebuje vsak vnos ve� stvari. Prvi podatek, ki je dolg dva bajta, vsebuje 
dol�ino niza, ki mu sledi, sledi mu niz, �tiri bajti za razvr��anje po skupinah ter dva bajta 
za dol�ino ostalih podatkov, ki pa so v obliki verige TLV paketov. Takih vnosov pa je mo�nih 
pribli�no 700 v tem paketu. Spra�ujem se, zakaj, �e protokol omogo�a hierarhi�no strukturo 
tega niso izkoristili.

Trenutna implementacija OSCARja v Jimmyju ni lepo realizirana, kajti lepa realizacija zahteva 
dobro specifikacijo protokola, katere za tale protokol ni. Odlo�il sem se, da bom uporabljeno 
dokumentacijo strnil v popolnej�o obliko ter dolo�ene podatke popravil in dopolnil razlage 
sekvenc. V Jimmyja bo potrebno vlo�iti �e marsikatero uro dela, za popravke, izbolj�ave 
ter vgradnjo dodatne funkcionalnosti.

Kar se protokola ti�e, je bil v osnovi dobro zami�ljen, a pritisk trga je z njim odigral grdo 
igro z vrivanjem nepremi�ljenih funkcionalnosti. Povozili so ga tudi rivali, ki uporabljajo 
novej�e principe kot je prenos informacije v obliki XML.

Protokol zaradi pre�iroke zasnove tako povzro�a velike redundance poslanih paketov. Veliko 
�tevilo neuporabljenih atributov in nesmiselnih podatkov se na tak na�in prena�a preko mre�e.
\chapter{Uporabni�ki vmesnik}

\section{Uvod}
Izvorne datoteke razredov uporabni�kega vmesnika se nahajajo v podimeniku ./ui, 
kateri se nahaja v korenskem imeniku z izvorno kodo. Imenik vsebuje naslednje datoteke:

\begin{itemize}
\item JimmyUI.java -- v tej datoteki se nahaja razred, ki je osnova za komunikacijo med 
uporabni�kim vmesnikom in ostalimi deli aplikacije.  Ves pretok informacij med uporabni�kim 
vmesnikom in razredi, ki skrbijo za protokol, poteka preko tega razreda.
\item MainMenu.java -- izvorna koda okna, kjer lahko uporabnik dodaja, bri�e in spreminja 
podatke o ra�unih za razli�ne protokole. Poleg tega lahko uporabnik na tej strani 
spro�i vzpostavljanje ali prekinitev povezave,
\item NewAccount.java -- podokno, kjer uporabnik dolo�i podrobnosti (uporabni�ko ime, 
geslo, ...) za ustvarjanje novega ali spremembo obstoje�ega ra�una,
\item ContactsMenu.java -- okno, kjer se izpi�e seznam kontaktov za vse ra�une, ki so 
trenutno povezani. Kontakti so razdeljeni v skupine;
\item EditContact.java -- podokno, kjer uporabnik dodaja nove ali spreminja obstoje�e kontakte;
\item ChatWindow.java -- okno za komunikacijo z izbranim uporabnikom;
\item About.java -- okno s podatki o avtorjih in osnovnih informacijah o programu;
\item Splash.java -- pozdravno okno, ki se prika�e ob zagonu programa in med operacijami, ko 
je program zaseden
\end{itemize}

Razen razreda JimmyUI, so vsi izpeljani (extendani) iz razli�nih implementacoij razreda 
Displayable. Vsak objekt tega tipa se lahko samostojno prika�e na zaslonu naprave kot 
samostojno okno. Posamezne implementacije slu�ijo specifi�nim problemom. Programer lahko 
za vsako okno izbere podrazred, ki mu za realizacijo dolo�enega okna najbolj ustreza. Na 
voljo imamo forme, sezname, tekstovna okna ipd.

Posamezna okna v Jimmy-ju sem implementiral tako, da sem jih izpeljal iz najprimernej�ih 
razredov Javine knji�nice in jim dodal �eljene funkcionalnosti. Ob zagonu programa se 
celoten uporabni�ki vmesnik ustvari kot objekt tipa JimmyUI, preko katerega se posredno 
ustvarijo vsa okna in gumbe.

Objekt JimmyUI skrbi za pravilno interpretiranje ukazov in komunikacijo med protokoli in 
uporabni�kim vmesnikom. �e �eli protokol posredovati informacijo v uporabni�ki vmesnik, 
to stori s klicem ustrezne metode iz vmesnika ProtocolInteraction.

\section{Opis dogajanja ob zagonu programa}
Ob zagonu programa se na zaslonu najprej prika�e pozdravno okno, ki uporabnika med procesom 
zaganjanja obve��a o trenutnem dogajanju. Temu sledi branje podatkov o ra�unih in  nastavitev 
iz pomnilnika. Zaradi  narave implementacije pomnilnika se nato izvede optimizacija spomina. 
Java podatke  shranjuje kot seznam zapisov. Problem je v tem, da brisanje zapisa dejansko ne 
izbri�e, ampak samo ozna�i kot neuporabnega. Med optimizacijo zato celoten spomin izbri�emo 
in ponovno shranimo podatke o ra�unih in nastavitvah. S tem se uspe�no znebimo neuporabnih 
zapisov v pomnilniku.

Od tu naprej imamo ve� mo�nosti:
\begin{enumerate}
\item �e ne obstaja �e noben ra�un, se uporabniku odpre okno za kreiranje novega ra�una;
\item v kolikor noben od obstoje�ih ra�unov ne zahteva avtomati�ne prijave, se odpre okno s 
seznamom ustvarjenih ra�unov, kjer uporabnik izbere ra�un s katerim se �eli prijaviti;
\item v primeru, da kateri od ra�unov zahteva avtomati�no prijavo, se ta izvede. Po uspe�no 
izvedeni prijavi se uporabniku prika�e okno s seznamom knotaktov.
\end{enumerate}

\section{Opis posameznih modulov uporabni�kega vmesnika}

V tem poglavju bom vsak del vmesnika opisal bolj podrobno. Opisane podrobnosti se nana�ajo 
na uporabni�ki vmesnik, kakr�nega prika�e Sunov emulator za J2ME aplikacije. Zaradi narave 
Javinih knji�nic za uporabni�ki vmesnik lahko na razli�nih napravah pride do razlik v prikazu 
in razporeditvi ukazov na zaslonu.


\subsection{Okno s seznamom aktivnih ra�unov}
V tem oknu so prikazani ustvarjeni uporabni�ki ra�uni. Poleg uporabni�kega imena je prikazan 
tudi logotip protokola (sli�ice so last avtorjev KDE progama Kopete). Uporabnik lahko v meniju 
izbere naslednje ukaze:

\begin{itemize}
\item Login -- povzro�i prijavo trenutno izbranega ra�una. Med povezovanjem se prika�e 
pozdravno okno. �e je prijava uspe�na, se prika�e seznam kontaktov, v nasprotnem primeru 
pa je uporabnik o neuspehu obve��en. Isto�asno je lahko povezanih ve� re�unov;
\item Logout -- povzro�i odjavo trenutno izbranega ra�una. Poleg tega se iz seznama 
odstranijo vsi kontakti, ki so pripadali tej povezavi;
\item New -- odpre okno za ustvarjanje novega ra�una;
\item Edit -- odpre okno za spreminjanje nastavitev izbranega ra�una;
\item Delete -- izbri�e izbrani ra�un;
\item Back -- zapre okno in se vrne na seznam kontaktov
\end{itemize}

\subsection{Okno za dodajanje novega ra�una}
Okno vsebuje naslednja polja:
\begin{itemize}
\item seznam protokolov;
\item polje za uporabni�ko ime. Uporabni�ko ime se vnese v obliki uporabjnik@domena. Polje 
je obvezno.
\item polje za geslo. Geslo se med vna�anjem zamaskira z zvezdicami. Polje je obvezno;
\item polje za ime stre�nika. �e se ne uporablja privzeti, uporabnik sem vnese ime 
nadomestnega stre�nika. Polje ni obvezno;
\item polje za �tevilko vrat. Uporablja se samo, �e se ne uporaljajo standardna vrata. Polje 
ni obvezno
\item izbira avtomati�ne prijave ob zagonu;
\end{itemize}

S klikom na gumb OK, se ra�un shrani in prika�e na seznamu obstoje�ih ra�unov. Back povzro�i 
brisanje vne�enih parametrov in vra�anje na seznam ra�unov.


\subsection{Okno za sreminjanje nastavitev obstoje�ega ra�una}
Okno vsebuje naslednja polja:
\begin{itemize}
\item geslo;
\item ime dodatnega stre�nika;
\item vrata;
\item izbira avtomati�ne prijave ob zagonu;
\end{itemize}

Okno je zelo podobno tistemu za dodajanje novega ra�una. V posameznih poljih se nahajajo 
trenutno nastavljene vrednosti, ki jih uporabnik lahko spremeni. Spremembe se shranijo s 
klikom na OK, pobri�ejo pa s klikom na Back. V obeh primerih se ponovno prika�e okno s 
seznamom ra�unov.

\subsection{Okno s seznamom kontakotv}
V tem oknu se nahaja seznam kontaktov vseh trenutno povezanih ra�unov. Kontakti so 
razdeljeni v skupine. Poleg vsakega kontakta je prikazan logotip protokola, 
kateremu pripada. Logotip je �rno-bel, �e ima kontakt status ``odsoten'',  bodisi barven 
v vseh drugih primerih. Uporabnik lahko v meniju izbere naslednje ukaze:
\begin{itemize}
\item Chat -- s klikom na ta gumb se pri�ne nov pogovor (ali nadaljuje obstoje�) s trenutno 
izbranim kontaktom. �e je trenutlo izbrana labela z imenom skupine se ne zgodi ni�.
\item New contact -- klik na ta gumb odpre pogovorno okno za dodajanje kontakta.
\item Delete contact  - klik na ta gumb izbri�e trnutno izbran kontakt.
\item Edit -- odpre okno za spreminjanje skupine in psevdonima za trenutno izbran kontakt.
\item Accounts -- odpre meni s seznamom ra�unov.
\item About -- prika�e podatke o avtrojih in programu.
\item Exit -- povzor�i izhod iz programa. Med uga�anjem programa se prika�e pozdravno okno.
\end{itemize}

\subsection{Okno za dodajanje kontakta}
Okno ima naslednja polja:
\begin{itemize}
\item Protokol -- v tem pop-up meniju izberemo protokol, kateremu �elimo dodati kontakt.
\item uporabni�ko ime -- sem vpi�emo uporabni�ko ime kontakta (oblike: uporabnik@domena)
\item psevdonim -- psevdonim, pod katerim se bo kontakt prikazal v seznamu. �e je polje 
prazno se bo prikazalo kar uporabni�ko ime
\item skupina -- v tem pop-up meniju izberemo bodisi edo izmed obstoje�ih skupin, bodisi 
uporabniku ne dodelimo nobene skupne (``No group''), bodisi ustvarimo novo (``Other'')
\item nova skupina -- �e smo v meniju za izbiranje skupine izbrali ``Other'', tukaj vpi�emo 
ime nove skupine. �e je polje prazno, se uporabniku ne dodeli nobena skupina.
\end{itemize}

S klikom na OK, se kontakt doda na seznam kontaktov. Poleg tega ustreznemu protokolu sporo�imo, 
da je bil dodan nov kontakt. S klikom na Back se vsebina okna pobri�e. V obeh primerih se 
vrnemo v okno s seznamom kontaktov.
�e nas kak drug uporabnik doda na svoj seznam kontaktov, protokol o tem obvestu uporabni�ki 
vmesknik in kontakt je avtomati�no dodan.

\subsection{Okno za spreminjanje kontaktov}
Okno ima naslednja polja:
\begin{itemize}
\item Psevdonim
\item skupina
\item nova skupina
\end{itemize}

Okno je zelo podobno tistemu za dodajanje kontaktov. Ob pritisku na OK se spremembe shranijo. 
O spremembah se obvesti ustrezen protokol. Pritisk na Back povzro�i brisanje vseh polj. V 
obeh primerih se vrnemo v seznam kontaktov.

\subsection{Okno za pogovor}
Okno vsebuje polje za vnos sporo�ila. Ob pritisku na Send se sporo�ilo po�lje. Ko prispe 
novo sporo�ilo od osebe, s katero se pogovarjamo, se avtomati�no prika�e ustrezno okno, 
kamor smo pred tem dodalo prispelo sporo�ilo.

\subsection{Pozdravno okno}
Okno prika�e Jimmy-jev logotip, kateremu se ob zagonu avtomati�no, glede na dimenzije 
zaslona, prilagodi velikost. Poleg logotipa se prika�e tudi sporo�ilo, ki podrobneje 
opisuje trenutno dogajanje vprogramu.
%-----------------------------------------------------------
\addcontentsline{toc}{chapter}{\numberline{4}Viri}
\begin{thebibliography}{9999}
%\bibitem[AC]{AC}A~Cottrell, {\sl Word Processors: Stupid and Inefficient},
%\\ \mbox{}\hfill{\tt http://www.ecn.wfu.edu/$\sim$cottrell/wp.html}
%\bibitem[URL]{URL} {\sl Example of a long URL},\\
%{\tt http://www-groups.dcs.st-andrews.ac.uk/$\sim$history/\dots
%\\ \mbox{}\hfill\dots Mathematicians/Newton.html}
%\bibitem[WDT]{WDT} {\sl WinEdt}, {\tt www.winedt.com}
%\bibitem[WSH]{WSH} {\sl WinShell}, {\tt www.winshell.de}
\bibitem[1]{Jabber1} {\sl Jabber-1},\\
{\tt http://www.jabber.org}
\bibitem[2]{Jabber2} {\sl Jabber-2},\\
{\tt http://www.ietf.org/rfc/rfc3920.txt}
\bibitem[3]{Jabber3} {\sl Jabber-3},\\
{\tt http://www.ietf.org/rfc/rfc3921.txt}
\bibitem[4]{Jabber4} {\sl Jabber-4},\\
{\tt http://bombus.jrudevels.org/}
\bibitem[5]{MSN1} {\sl MSN-1},\\
{\tt http://www.hypothetic.org/docs/msn/index.php}
\bibitem[6]{MSN2} {\sl MSN-2},\\
{\tt http://msnpiki.msnfanatic.com/index.php/Main\_Page}
\bibitem[7]{ICQ1} {\sl ICQ-1},\\
{\tt http://www.micq.org/ICQ-OSCAR-Protocol-v7-v8-v9/index.html}
\bibitem[8]{ICQ2} {\sl ICQ-2},\\
{\tt http://iserverd.khstu.ru/oscar/}
\bibitem[9]{ICQ3} {\sl ICQ-3},\\
{\tt http://www.oilcan.org/oscar/}
\bibitem[10]{ICQ4} {\sl ICQ-4},\\
{\tt http://www.kingant.net/oscar/}
\end{thebibliography}
%-----------------------------------------------------------
%\appendix
%\include{app1}
%\include{app2}
%-----------------------------------------------------------
\end{document}
